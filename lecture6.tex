%
% This is the LaTeX template file for lecture notes for EE 382C/EE 361C.
%
% To familiarize yourself with this template, the body contains
% some examples of its use.  Look them over.  Then you can
% run LaTeX on this file.  After you have LaTeXed this file then
% you can look over the result either by printing it out with
% dvips or using xdvi.
%
% This template is based on the template for Prof. Sinclair's CS 270.

\documentclass[twoside]{article}
\usepackage{graphics}
\setlength{\oddsidemargin}{0.25 in}
\setlength{\evensidemargin}{-0.25 in}
\setlength{\topmargin}{-0.6 in}
\setlength{\textwidth}{6.5 in}
\setlength{\textheight}{8.5 in}
\setlength{\headsep}{0.75 in}
\setlength{\parindent}{0 in}
\setlength{\parskip}{0.1 in}

%
% The following commands set up the lecnum (lecture number)
% counter and make various numbering schemes work relative
% to the lecture number.
%
\newcounter{lecnum}
\renewcommand{\thepage}{\thelecnum-\arabic{page}}
\renewcommand{\thesection}{\thelecnum.\arabic{section}}
\renewcommand{\theequation}{\thelecnum.\arabic{equation}}
\renewcommand{\thefigure}{\thelecnum.\arabic{figure}}
\renewcommand{\thetable}{\thelecnum.\arabic{table}}

%
% The following macro is used to generate the header.
%
\newcommand{\lecture}[4]{
   \pagestyle{myheadings}
   \thispagestyle{plain}
   \newpage
   \setcounter{lecnum}{#1}
   \setcounter{page}{1}
   \noindent
   \begin{center}
   \framebox{
      \vbox{\vspace{2mm}
    \hbox to 6.28in { {\bf EE 382C/361C: Multicore Computing
                        \hfill Fall 2016} }
       \vspace{4mm}
       \hbox to 6.28in { {\Large \hfill Lecture #1: #2  \hfill} }
       \vspace{2mm}
       \hbox to 6.28in { {\it Lecturer: #3 \hfill Scribe: #4} }
      \vspace{2mm}}
   }
   \end{center}
   \markboth{Lecture #1: #2}{Lecture #1: #2}
   %{\bf Disclaimer}: {\it These notes have not been subjected to the
   %usual scrutiny reserved for formal publications.  They may be distributed
   %outside this class only with the permission of the Instructor.}
   \vspace*{4mm}
}

%
% Convention for citations is authors' initials followed by the year.
% For example, to cite a paper by Leighton and Maggs you would type
% \cite{LM89}, and to cite a paper by Strassen you would type \cite{S69}.
% (To avoid bibliography problems, for now we redefine the \cite command.)
% Also commands that create a suitable format for the reference list.
\renewcommand{\cite}[1]{[#1]}
\def\beginrefs{\begin{list}%
        {[\arabic{equation}]}{\usecounter{equation}
         \setlength{\leftmargin}{2.0truecm}\setlength{\labelsep}{0.4truecm}%
         \setlength{\labelwidth}{1.6truecm}}}
\def\endrefs{\end{list}}
\def\bibentry#1{\item[\hbox{[#1]}]}

%Use this command for a figure; it puts a figure in wherever you want it.
%usage: \fig{NUMBER}{SPACE-IN-INCHES}{CAPTION}
\newcommand{\fig}[3]{
			\vspace{#2}
			\begin{center}
			Figure \thelecnum.#1:~#3
			\end{center}
	}
% Use these for theorems, lemmas, proofs, etc.
\newtheorem{theorem}{Theorem}[lecnum]
\newtheorem{lemma}[theorem]{Lemma}
\newtheorem{proposition}[theorem]{Proposition}
\newtheorem{claim}[theorem]{Claim}
\newtheorem{corollary}[theorem]{Corollary}
\newtheorem{definition}[theorem]{Definition}
\newenvironment{proof}{{\bf Proof:}}{\hfill\rule{2mm}{2mm}}

% **** IF YOU WANT TO DEFINE ADDITIONAL MACROS FOR YOURSELF, PUT THEM HERE:
\usepackage{listings}
\usepackage{color}

\definecolor{dkgreen}{rgb}{0,0.6,0}
\definecolor{gray}{rgb}{0.5,0.5,0.5}
\definecolor{mauve}{rgb}{0.58,0,0.82}

\lstset{frame=tb,
  language=Java,
  aboveskip=3mm,
  belowskip=3mm,
  showstringspaces=false,
  columns=flexible,
  basicstyle={\small\ttfamily},
  numbers=none,
  numberstyle=\tiny\color{gray},
  keywordstyle=\color{blue},
  commentstyle=\color{dkgreen},
  stringstyle=\color{mauve},
  breaklines=true,
  breakatwhitespace=true,
  tabsize=3
}
\graphicspath{ {/} }
\begin{document}
%FILL IN THE RIGHT INFO.
%\lecture{**LECTURE-NUMBER**}{**DATE**}{**LECTURER**}{**SCRIBE**}
\lecture{6}{September 08}{Vijay Garg}{Jiaolong Yu}
%\footnotetext{These notes are partially based on those of Nigel Mansell.}

% **** YOUR NOTES GO HERE:

% Some general latex examples and examples making use of the
% macros follow.  
%**** IN GENERAL, BE BRIEF. LONG SCRIBE NOTES, NO MATTER HOW WELL WRITTEN,
%**** ARE NEVER READ BY ANYBODY.
\section{Lower Bound on the Number of Shared Memory Location}
\begin{theorem}
(Burns and Lynch)Any mutex algorithem that uses only RW on n processes require at least n shared locations.
\end{theorem}

\begin{proof}
Consider 2 processes, say P and Q, in competing for Critical Section. They have only one shared memory location A. 
Let Q run till it's about to write to A. 
Let P run and enter critical section.
Let Q run again. It enters CS.
Mutex violation.
Consider 3 processes, say P, Q and R. They have 2 shared memory locations A and B.
Let P and Q run till they are about to write to A.
Let R run and enter critical section.
let P and Q run again. One of them will enter critcal section.
Mutex violation.
With this method we can extend the situation to n processors and prove the theorem.
\end{proof}

\begin{definition}
Covering state. All share variables are about to be overwritten by processes and the shared stats is consistant with no process in CS.
\end{definition}
\section{Fischer's Algorithem}

Turn = -1 Means the door is open.\newline
\textbf{RequestCS:}
\begin{lstlisting}
while(ture) {
  while(turn != -1);
  turn = i;
  wait_for_delta_time_units();
  if(turn == i) return;
}
\end{lstlisting}
\textbf{ReleaseCS:}
\begin{lstlisting}
turn = -1
\end{lstlisting}

This algorithm can cause mutex violation. One senario:\newline
P_i$ reads turn = -1$\newline
P_j$ reads turn = -1$\newline
P_j$ sets turn = j$\newline
P_j$ reads turn = j$ and enter CS\newline
P_i$ sets turn = i$\newline
P_i$ reads turn = i and enter CS$\newline

\begin{theorem}
Assuming $delta >= C$, Fischer's Algorithm satisfies mutex. C: maximum time required to close the door, ie. set the turn.
\end{theorem}

\begin{proof}
Let P_i$ be the processor that enters CS successfully. Assume there is another processor $P_j$ that may enter CS.$\newline
  \includegraphics{proof0}\newline
P_j$ must writes turn after $P_i$ reads it to be -1. If it writes before $P_i$ writes turn, after waiting for delta units of time which is longer than C, then $P_i$ must have already finished writing turn, $P_j$ will defnitely read turn = i then it will not enter CS. If $P_j$ writes turn after $P_i$ writes turn, then $P_i$ will read turn = j after waiting delta units of time, then it will not enter CS. Proved that only one processor will enter CS$
\end{proof}

\section{Lamport's Fast Algorithm}
This algorithm enables for processors to enter CS fast when there is no contention. \newline
No contention -- fast path
Contention -- slow path
\subsection{Splitter}
\includegraphics{splitter}
\begin{lstlisting}
For every P_i:
  Variables: 
    door: {open, closed}, initially open;
    last: pid, initially -1;
  last = i;
  if (door == closed) return left;
  else {
    door = closed;
    if (last == i) return down;
    else return right;
  }
\end{lstlisting}
\begin{claim}
  $|left| <= n-1$
\end{claim}
\begin{proof}
  Someone need to close the door
\end{proof}
\begin{claim}
  $|right|<=n-1$
\end{claim}
\begin{proof}
  Consider processor P_i$ such that last = i then $P_i$ is part of left or down. So as at least one process would not go right.$
\end{proof}
\begin{claim}
  $|down| <= 1$
\end{claim}
\begin{proof}
  Let P_i$ be the first process to go down. Assume there is another processor $P_j$ that may go down.$\newline
  \includegraphics{proof.png}\newline
  $As shown in the picture, if $P_j$ does everything bofore $P_i$ writes last, then $P_j$ would be the first one to go down, which conflicts with our assumption. However, $P_j$ must writes to last before $P_i$ does otherwise $P_i$ would not be able to read last = i at the end. $P_j$ must closes the door after $P_i$ checks the door, otherwise $P_i$ would have found the door closed, then it cannot go down. Then we have $P_i$ writes to last before $P_i$ checks the door, before $P_j$ closes the door, before $P_j$ reads last. It means $P_j$ must reads last = i then it can not go down.$
\end{proof}

\subsection{Lamport's Fast Algorithm}
\includegraphics{fast}
\newline
\textbf{RequestCS:}
\begin{lstlisting}
while (ture) {
  flag[i] = up;
  x = i;
  if (y != -1) { // split left
    flag[i] = down;
    waituntil( y == -1);
    continue;
  } else {
    y = i;
    if (x == i) return; // down
    else { // right
      flag[i] = down;
      waituntil(y = -1);
      continue;
    }
  }
}
\end{lstlisting}
\textbf{ReleaseCS:}
\begin{lstlisting}
y = -1;
flag[i] = down;
\end{lstlisting}
\end{document}